\documentclass{article}
\usepackage[utf8]{inputenc}
\usepackage[portuguese]{babel}
\usepackage[a4paper, total={7in, 9in}]{geometry}
\usepackage{graphicx}
\usepackage{float}
\usepackage[title]{appendix}
\usepackage[style=numeric]{biblatex}
\usepackage{csquotes}
\usepackage{mathtools}
\usepackage{xcolor}
%\usepackage{minted}
%\usepackage{framed}
%\addbibresource{references.bib}

\begin{document}

{
\center
\begin{figure}[H]
        \centering
        \includegraphics[width=3cm]{Pictures/UM_EENG.jpg}
\end{figure}
\textsc{\Large Universidade do Minho} \\ [0.5cm]
\textsc{\Large Mestrado em Engenharia Informática} \\ [0.5cm]
\textsc{\large Processamento e Representação de Conhecimento} \\ [0.5cm]

{\LARGE \bfseries Musike - Music Linked} \\[0.5cm]

\begin{tabular}{c} 
    José Carlos Lima Martins \\
    A78821 \\
\end{tabular} \\[0.5cm]

\today \\[1cm]
}

\section{Introdução}

ENTREGA:\\
27 de Maio\\

\section{Contextualização}

O \textbf{Musike}, abreviatura de \textit{Music Linked}, consiste num website com a informação de artistas, bem como, as suas músicas e albuns. Em termos de informação a apresentar, pretende-se o seguinte:
\begin{itemize}
    \item Artista
        \begin{itemize}
            \item nome
            \item alias
            \item tipo de artista (Grupo, Pessoa, etc)
            \item data de nascimento/inicio
            \item data de falecimento/fim
            \item sexo (não é aplicável a todos os tipos de artista)
            \item nacionalidade (obtido através da área)
            \item descrição
            \item relações com outros artistas, com músicas e com albuns
            \item urls para página pessoal, redes sociais, etc
            \item classificação dos utilizadores do website
            \item soma das visualizações das várias músicas
        \end{itemize}
    \item Album
        \begin{itemize}
            \item título
            \item data do primeiro release
            \item artista(s)
            \item descrição
            \item músicas do album
            \item tipos do album
            \item relações com artistas e com outros albuns
            \item urls sobre o album
            \item classificação dos utilizadores do website
            \item soma das visualizações das várias músicas
        \end{itemize}
    \item Música
        \begin{itemize}
            \item título
            \item artista(s)
            \item duração
            \item descrição
            \item língua(s)
            \item tipos da música (clássica, rock, etc)
            \item relações com outras músicas e com artista
            \item urls sobre a música
            \item classificação dos utilizadores do website
            \item número de vezes ouvida pelos utilizadores do website
        \end{itemize}
    \item Área
        \begin{itemize}
            \item nome
            \item tipo (país, cidade, etc)
            \item alias
            \item data de creação
            \item data de fim
            \item descrição
            \item urls sobre a área
            \item relações com outas áreas, visto que uma área pode ser parte de outra e uma área pode ter várias áreas
        \end{itemize}
\end{itemize}

Para além disso, para cada música o objetivo é ter também o video do youtube embutido bem como as lyrics e por cada visualização do video contar que a música foi ouvida uma vez.

Por outro lado, o website deve permitir aos utilizadores escolher as músicas e albums que mais gosta, bem como, puder criar playlists.

Cada utilizador deve ter acesso às suas estatisticas, que deve ter as músicas que o utilizador ouve mais, os artistas que ouve mais e os albuns que ouve mais.

De forma adicional, seria interessante dar sugestões ao utilizador de músicas a ouvir.

Por fim, apresentar estatísticas gerais do website entre as quais:
\begin{itemize}
    \item músicas mais ouvidas pelos utilizadores
    \item artistas mais ouvidos pelos utilizadores
    \item tipos de músicas (tags) mais ouvidos pelos utilizadores
    \item albuns mais ouvidos pelos utilizadores
    \item paises com mais artistas
    \item paises com mais músicas
    \item paises com mais albuns
    \item paises mais ``ouvidos'' pelos utilizadores
\end{itemize}

\section{Ontologia}

TODO: explicar a estrutura da ontologia

\section{De JSON para RDF-Turtle}

Com a estrutura da ontologia e com uma ideia de que informação seria necessária, foi então usado um dataset (\url{http://ftp.musicbrainz.org/pub/musicbrainz/data/json-dumps/20190403-001001/}) para popular a ontologia. Este dataset é proveniente do \textbf{MusicBrainz} estando o mesmo em JSON. O seu timestamp é de 2019\-04\-03, portanto um dataset bastante recente possuindo cerca de 240GB de tamanho. Visto o mesmo estar em JSON é necessário então convertê-lo para Turtle.

Num primeiro passo, realizou-se correções a este dataset, visto que bastantes identificadores de entidades estavam desatualizados, não permitindo uma correta conversão da informação. Foi então criado uma script para tal efeito.

TODO: explicar aqui as correções feitas aos datasets e os conversores

\section{Conclusão}

\newpage 
\printbibliography

\begin{appendices}

\end{appendices}

\end{document}
