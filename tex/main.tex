\documentclass{article}
\usepackage[utf8]{inputenc}
\usepackage[portuguese]{babel}
\usepackage[a4paper, total={7in, 9in}]{geometry}
\usepackage{graphicx}
\usepackage{float}
\usepackage[title]{appendix}
\usepackage[style=numeric]{biblatex}
\usepackage{csquotes}
\usepackage{mathtools}
\usepackage{xcolor}
%\usepackage{minted}
%\usepackage{framed}
%\addbibresource{references.bib}

\begin{document}

{
\center
\begin{figure}[H]
        \centering
        \includegraphics[width=3cm]{Pictures/UM_EENG.jpg}
\end{figure}
\textsc{\Large Universidade do Minho} \\ [0.5cm]
\textsc{\Large Mestrado em Engenharia Informática} \\ [0.5cm]
\textsc{\large Processamento e Representação de Conhecimento} \\ [0.5cm]

{\LARGE \bfseries Musike - Music Linked} \\[0.5cm]

\begin{tabular}{c} 
    José Carlos Lima Martins \\
    A78821 \\
\end{tabular} \\[0.5cm]

\today \\[1cm]
}

\section{Introdução}

ENTREGA:\\
27 de Maio\\

\section{Contextualização}

O trabalho a desenvolver passa por escolher uma área/tema de trabalho. Com esse tema em vista, criar/construir/limpar/extrair um dataset com informação envolvendo tal tema e que esteja disponível na LOD (\textit{Linked Open Data}). Este dataset em Turtle (\textit{Terse RDF Triple Language}) será armazenado em \textit{GraphDB}, onde será possível aceder/realizar queries ao dataset localmente.

De seguida, será criado um website/app web de forma a explorar o dataset anteriormente criado. Este website deve possuir autenticação, usando para isso o \textit{MongoDB}.

\section{Tema escolhido - \textbf{Musike}}

O \textbf{Musike}, abreviatura de \textit{Music Linked}, consiste num website com a informação de artistas, bem como, das suas músicas e dos seus albuns. Em termos de informação a apresentar, pretende-se o seguinte:
\begin{itemize}
    \item Artista
        \begin{itemize}
            \item nome
            \item alias
            \item tipo de artista (Grupo, Pessoa, etc)
            \item data de nascimento/inicio
            \item data de falecimento/fim
            \item sexo (não é aplicável a todos os tipos de artista)
            \item nacionalidade (obtido através da área)
            \item descrição
            \item relações com outros artistas, com músicas e com albuns
            \item urls para página pessoal, redes sociais, etc
            \item classificação dos utilizadores do website
            \item soma das visualizações das várias músicas
        \end{itemize}
    \item Album
        \begin{itemize}
            \item título
            \item data do primeiro release
            \item artista(s)
            \item descrição
            \item músicas do album
            \item tipos (tags) do album (clássica, rock, etc)
            \item relações com artistas e com outros albuns
            \item urls sobre o album
            \item classificação dos utilizadores do website
            \item soma das visualizações das várias músicas
        \end{itemize}
    \item Música
        \begin{itemize}
            \item título
            \item artista(s)
            \item duração
            \item descrição
            \item língua(s)
            \item tipos (tags) da música (clássica, rock, etc)
            \item relações com outras músicas e com artista
            \item urls sobre a música
            \item classificação dos utilizadores do website
            \item número de vezes ouvida pelos utilizadores do website
        \end{itemize}
    \item Área
        \begin{itemize}
            \item nome
            \item tipo (país, cidade, etc)
            \item alias
            \item data de creação
            \item data de fim
            \item descrição
            \item urls sobre a área
            \item relações com outas áreas, pois uma área pode ser parte de outra e, vice-versa, uma área pode incluir várias áreas
        \end{itemize}
\end{itemize}

Para além disso, para cada música o objetivo é ter na sua página o video presente no \textit{YouTube} bem como a letra da música. Com isto, pretende-se que por cada visualização do video se conte que a música foi ouvida uma vez.

Por outro lado, o website deve permitir aos utilizadores escolher as músicas e albuns que mais gosta, bem como, puder criar playlists.

Cada utilizador deve ter acesso às suas estatisticas, onde deve estar presente as músicas que o utilizador ouve mais, os artistas que ouve mais e os albuns que ouve mais.

Por fim, apresentar estatísticas gerais do website entre as quais:
\begin{itemize}
    \item músicas mais ouvidas pelos utilizadores
    \item artistas mais ouvidos pelos utilizadores
    \item tipos de músicas (tags) mais ouvidos pelos utilizadores
    \item albuns mais ouvidos pelos utilizadores
    \item paises com mais artistas
    \item paises com mais músicas
    \item paises com mais albuns
    \item paises mais ``ouvidos'' pelos utilizadores
\end{itemize}

De forma adicional, seria interessante dar sugestões ao utilizador de músicas a ouvir a partir das estatisticas do utilizador.

\section{Parte Estrutural da Ontologia}

Portanto, a partir do que se pretende construir e tendo já em ideia o dataset a ser usado (\textit{MusicBrainz}) foi construída a seguinte estrutura:

\begin{figure}[H]
        \centering
        \includegraphics[width=15cm]{Pictures/ontologySctructure.png}
\end{figure}

As classes principais são \textit{Artist}, \textit{Recording}, \textit{Album} e \textit{Area} que representam respetivamente Artista, Música, Album e Área. Uma classe também importante mas secundária, é \textit{URL} que reprensenta URL's. Por forma a organizar melhor as propriedades, visto que as subclasses herdam as propriedades/relações das superclasses, foram criadas as superclasses \textit{Entity}, \textit{MusicEntity} e \textit{TimeLimitedEntity}. \textit{MusicEntity} inclui as subclasses \textit{Recording} e \textit{Album} devido às duas possuirem a propriedade \textit{title}. Por outro lado, \textit{TimeLimitedEntity} tem como subclasses \textit{Artist} e \textit{Area}, visto que estas duas classes possuem uma data de inicio e de fim. Já a classe \textit{Entity} inclui as subclasses \textit{MusicEntity} e \textit{TimeLimitedEntity}, ou seja, é a superclasse das quatro principais classes, até porque estas quatro classes possuem as propriedades \textit{about} e \textit{disambiguation}. Existe ainda uma terceira classe, \textit{Relation}, onde se pretende representar as várias relações que existem entre Entidades (classe \textit{Entity}).

Quanto às propriedades/relações a ontologia apresenta as seguintes:
\begin{itemize}
    \item Propriedades (\textit{Data Properties})
    \begin{itemize}
        \item Classe \textbf{\textit{Entity}}
        \begin{itemize}
            \item \textbf{\textit{about}}: Descrição do elemento
            \item \textbf{\textit{disambiguation}}: Forma de disambiguar entre elementos com o mesmo nome
        \end{itemize}
        \item Classe \textbf{\textit{Relation}}
        \begin{itemize}
            \item \textbf{\textit{relationType}}: o nome da relação entre duas entidades (classe \textit{Entity})
        \end{itemize}
        \item Classe \textbf{\textit{URL}}
        \begin{itemize}
            \item \textbf{\textit{label}}: o nome do website a que se refere o URL (exe: wikipédia)
            \item \textbf{\textit{value}}: URL propriamente dito
        \end{itemize}
        \item Classe \textbf{\textit{TimeLimitedEntity}}
        \begin{itemize}
            \item \textbf{\textit{name}}: Nome do elemento
            \item \textbf{\textit{alias}}: nomes alternativos e nomes com erros ortográficos, permitindo uma melhor busca quando o utilizador introduz o nome com erros 
            \item \textbf{\textit{type}}: Tipo do elemento (em \textit{Area}: district, contry, city, etc, já em \textit{Artist}: person, group, choir, etc)
            \item \textbf{\textit{beginDate}}: Data de ínicio
            \item \textbf{\textit{endDate}}: Data de fim
        \end{itemize}
        \item Classe \textbf{\textit{MusicEntity}}
        \begin{itemize}
            \item \textbf{\textit{title}}: Título
        \end{itemize}
        \item Classe \textbf{\textit{Album}}
        \begin{itemize}
            \item \textbf{\textit{firstReleaseDate}}: Data da primeira ``release'' do Album
        \end{itemize}
        \item Classe \textbf{\textit{Recording}}
        \begin{itemize}
            \item \textbf{\textit{duration}}: Duração da música
            \item \textbf{\textit{language}}: Língua da música
            \item \textbf{\textit{tag}}: Tags (tipos) da música
        \end{itemize}
        \item Classe \textbf{\textit{Artist}}
        \begin{itemize}
            \item \textbf{\textit{gender}}: Género do artista
            \item \textbf{\textit{sortName}}: Nome de modo a ordenar o artista numa lista
        \end{itemize}
    \end{itemize}
    \item Relações (\textit{Object Properties})
    \begin{itemize}
        \item \textbf{\textit{from}}: De \textit{Arist} para \textit{Area}, indica que um artista é da área
        \item \textbf{\textit{hasURL}}: De \textit{Entity} para \textit{URL}, indica que uma entidade tem o URL
        \item \textbf{\textit{domain}}: De \textit{Relation} para \textit{Entity}, indica o domínio (URI) da relação
        \item \textbf{\textit{range}}: De \textit{Relation} para \textit{Entity}, indica o contradomínio (URI) da relação
        \item \textbf{\textit{hasPart}}: De \textit{Area} para \textit{Area}, indica que uma área inclui a outra
        \item \textbf{\textit{partOf}}: Inverso de \textit{hasPart}, indica que uma área é parte de outra
        \item \textbf{\textit{artistCredit}}: De \textit{Recording} para \textit{Artist}, indica que a música tem como crédito o artista
        \item \textbf{\textit{hasCreditIn}}: Inverso de \textit{artistCredit}, indica que um artista tem crédito na música
        \item \textbf{\textit{hasTrack}}: De \textit{Album} para \textit{Recording}, indica que um album tem a música
        \item \textbf{\textit{trackIn}}: Inverso de \textit{hasTrack}, indica que uma música pertence a um album
    \end{itemize}
\end{itemize}

É importante voltar a referir que as subclasses herdam as propriedades das superclasses.

\section{De JSON para Turtle - Povoamento da Ontologia}

Com a estrutura da ontologia e com uma ideia de que informação seria necessária, foi então usado um dataset (\url{http://ftp.musicbrainz.org/pub/musicbrainz/data/json-dumps/20190403-001001/}) para popular a ontologia. Este dataset é proveniente do \textbf{MusicBrainz} estando o mesmo em JSON. O seu timestamp é de 2019-04-03, portanto um dataset bastante recente possuindo cerca de 240GB de tamanho. Visto o mesmo estar em JSON é necessário então convertê-lo para Turtle.

Num primeiro passo, percorre-se os ficheiros JSON com os conversores criados em \textit{Node.js}.

Num segundo passo, realizou-se correções a este dataset, visto que bastantes identificadores de entidades estavam desatualizados, resultando em informação mal associada. Este passo foi realizado depois da conversão em vez de antes, visto que o resultado final manter-se-á igual e como existe menos dados a percorrer, a correção é muito mais rápida. Foi então criado três scripts para tal efeito, sendo os mesmos dependentes entre si de forma encadeada. O script inicial, acede ao website do \textbf{MusicBrainz} para cada identificador referindo o tipo de entidade, visto que o roteamento usado pelo \textbf{MusicBrainz} é o seguinte \url{https://musicbrainz.org/NomeDaEntidade/identificador}. Ao aceder a cada página, caso o id tenha sido movido é devolvido uma página dizendo qual o novo identificador, caso não exista, devolve que a página não existe, e por fim, caso o identificador esteja correto devolve a página do mesmo. Este script criado recebe como argumento \textit{NomeDaEntidade/identificador} e devolve OK caso a página exista, null caso não exista e um identificador (o novo) se tiver sido movido. Depois esta script é usada por uma em \textit{Node.js}que procura pelos ids, chama a script anteriormente referida e substitui conforme o resultado. Esta script em \textit{Node.js}, lê um ficheiro passado como argumento, linha a linha, de forma a puder ler ficheiros grandes sem necessitar de muitos recursos, e devolve cada linha com as substituições para o \text{stdout}. Ainda em relação a esta script, é guardado num objeto os identificadores já verificados e a substituição a realizar (seja o mesmo id ou diferente), para que no caso de aparecerem nas linhas seguintes, não ser necessário aceder ao website do \textbf{MusicBrainz}, visto ser custoso em termos temporais. Por fim, devido ao script ser relativamente lento devido aos constantes acessos à \textit{Internet}, como já referido, de forma a acelerar o processamento, foi criada outra script que usa as duas anteriores. Esta última script divide o ficheiro passado como argumento em 6 ficheiros (usando o split) com o mesmo número de linhas. Depois aplica o script em \textit{Node.js} a cada ficheiro de forma paralela, ou seja, cria 6 processos concorrentes, redirecioando o output para um ficheiro (um ficheiro diferente por processo). Quando todos os processos tiverem terminado junta os ficheiros, usando o \textit{cat}. Por fim, remove os ficheiros auxiliares criados que já não são necessários.

TODO: explicar aqui as correções feitas aos datasets e os conversores

\section{Conclusão}

\newpage 
\printbibliography

\begin{appendices}

\end{appendices}

\end{document}
